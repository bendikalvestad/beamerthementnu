\documentclass[aspectratio=169]{beamer}
\usepackage[english]{babel}
\usepackage{xcolor}
\usepackage{transparent}
\usepackage[default,scale=1]{opensans}
\usepackage[T1]{fontenc}
\usepackage[utf8]{inputenc}

\mode<presentation>
{
	\usefonttheme{structurebold}  % or try serif, structurebold, ...
%Uncomment the following to display navigation symbols
	\setbeamertemplate{navigation symbols}{}
	%Constructing the frame title
	\setbeamertemplate{frametitle}{% 
		\kern1em\hskip-15pt
		\usebeamercolor[fg]{section}% 
		\usebeamerfont{section}% 
		\insertsection \hspace{0,1em} - {\normalsize \insertframetitle}
	}
	%Constructing the footline
	\setbeamertemplate{footline}{% 
		\kern1em\hskip3em% 
		\includegraphics[width=0.3\textwidth]{ntnulogo_eng.png}
		\hfill% 
		\usebeamercolor[fg]{page number in head/foot}% 
		\usebeamerfont{page number in head/foot}% 
		\insertframenumber%
%Uncomment the following line to display the total number of pages in the footnote
		%\,/\,\inserttotalframenumber
		\hskip12pt%
		\kern1.5em\vskip2em% 
}
	%Defining fonts
	\setbeamerfont{title}{shape=\bfseries, size=\huge}
	\setbeamerfont{subtitle}{series=\mdseries,size=\Large}
	
	%Defining the colors. Here you can find more elements whose color can be modified:
	%http://www.cpt.univ-mrs.fr/~masson/latex/Beamer-appearance-cheat-sheet.pdf
	\definecolor{NTNUBlue}{HTML}{00509e}
	\definecolor{LightGrey}{HTML}{D3D3D3}
	\setbeamercolor{background canvas}{bg=NTNUBlue}
	\setbeamercolor{title}{fg=white}
	\setbeamercolor{subtitle}{fg=white}
	\setbeamercolor{date}{fg=LightGrey}
	\setbeamercolor{author}{fg=LightGrey}
	\setbeamercolor{frametitle}{fg=NTNUBlue}
	\setbeamercolor{itemize item}{fg=NTNUBlue}
	\setbeamercolor{enumerate item}{fg=NTNUBlue}
	\setbeamercolor{block title}{fg=NTNUBlue}
	\setbeamercolor{itemize subitem}{fg=NTNUBlue}
	\setbeamercolor{enumerate subitem}{fg=NTNUBlue}
} 
	\usepackage{hyperref}
	\hypersetup{
	colorlinks=true,% make the links colored
	linkcolor=NTNUBlue,
	urlcolor=NTNUBlue
	}
	% Enforcing the final page
	\AtEndDocument{\begin{frame}[plain, noframenumbering]
		\begin{center}
			\vspace{4em}
			{\huge Thank you for your attention}\\
			\vspace{5em}
			\includegraphics[width=0.7\textwidth]{ntnulogo_eng.png}
		\end{center}
	\end{frame}
	}

% ---------->	Write here the content of the front page <----------
	\title[Your Short Title]{Front page/ put your title here}
	\subtitle{Subtitle. Author/date/year}
	\institute{\includegraphics[width=0.7\textwidth]{ntnulogo_eng_neg.png}}
	%Remember to uncomment the 3 lines in the next section in order to dsplay the author
	\author{Name of the author}
	%Remember to uncomment the 3 lines in the next section in order to dsplay the date
	\date{\today}

\begin{document}
	
	%----------------------------------Constructing the front page----------------------------------
	\begin{frame}[plain, noframenumbering]
		\vfill
		\centering	
		\begin{beamercolorbox}[sep=8pt,center,colsep=-4bp,rounded=true,shadow=true]{institute}
			\usebeamerfont{institute}\insertinstitute
		\end{beamercolorbox}	
		\vskip2.5em\par
		{\usebeamercolor[fg]{titlegraphic}\inserttitlegraphic\par}	
		\begin{beamercolorbox}[sep=8pt,center,colsep=-4bp,rounded=true,shadow=true]{title}
			\usebeamerfont{title}\MakeUppercase{\inserttitle}\par%
			\ifx\insertsubtitle\@empty%
			\else%
			\vskip0.5em%
			{\usebeamerfont{subtitle}\usebeamercolor[fg]{subtitle}\insertsubtitle\par}%
			\fi%     
		\end{beamercolorbox}%
		
% ---------->	Uncomment the following to display author information <----------
		%	\begin{beamercolorbox}[sep=8pt,center,colsep=-4bp,rounded=true,shadow=true]{author}
		%		\usebeamerfont{author}\insertauthor
		%	\end{beamercolorbox}
		
% ---------->	Uncomment the following to display date <----------
		%	\begin{beamercolorbox}[sep=8pt,center,colsep=-4bp,rounded=true,shadow=true]{date}
		%		\usebeamerfont{date}\insertdate
		%	\end{beamercolorbox}\vskip0.5em

% ---------->	If you add author or date, reduce the following space <----------
		\vskip4em
		
	\end{frame}
	
	%Setting the background color to none
	\setbeamercolor{background canvas}{bg=}
	
% ---------->	Uncomment these lines for an automatically generated table of contents. <----------
	%\begin{frame}{Outline}
	%  \tableofcontents
	%\end{frame}
	
	%----------------------------------Example section: Introduction----------------------------------
	\section{Introduction}
	
	\begin{frame}{A typical slide}
		
		
		\begin{itemize}
			\item Write here the introduction
			\item Use \texttt{itemize} or \texttt{enumerate} to organize your main points.
		\end{itemize}
				
		\begin{block}{Example}
			Rather than freely writing text in the frame, the use of blocks (such as this) is recommended.
			Here you can find a short introduction to the basis of beamer:
			\begin{enumerate}
				\item https://www.overleaf.com/learn/latex/beamer
			\end{enumerate}
		\end{block}
		
	\end{frame}	
	
\end{document}
